%TCIDATA{LaTeXparent=0,0,relatorio.tex}

\resumo{Resumo}{Escrever solu��es para problemas computacionais envolve um componente individual, cada pessoa possui uma maneira pr�pria de construir algoritmos em uma determinada linguagem de programa��o. Este tipo de conduta pode causar problemas quando h� v�rias pessoas trabalhando no mesmo projeto. Uma forma de contornar esta dificuldade � adotar um estilo de codifica��o para a equipe. Por�m, em projetos \textit{opensource}, torna-se dif�cil o uso de mecanismos que forcem a ado��o de um estilo de codifica��o, de modo que a adequa��o ou n�o ao estilo escolhido � uma op��o individual de cada desenvolvedor . Este trabalho como objetivo mensurar a ado��o do estilo de codifica��o PEP8 da linhguagem Python em projetos de c�digo livre. Para tal fim, foram analisados aproximadamente 62,000 projetos oriundos da base PyPI. Foram contabilizadas as viola��es do estilo a n�vel de projetos e m�dulos, e a partir dos resultados obtidos, foi discutido o n�vel de ado��o do estilo em quest�o.
}

\pagebreak

\resumo{Abstract}{Write solutions to computational problems involving an individual component, each person has their own way to build algorithms in a given programming language. This type of conduct can cause problems when there are several people working on the same project. One way around this difficulty is to adopt a coding style to the team. However, in projects opensource, it becomes difficult to use mechanisms that force the adoption of a coding style, so that the adequacy or not the chosen style is an individual choice of each developer. This study aimed to measure the adoption of PEP8 coding style of Python linhguagem in open source projects. To this end, they analyzed approximately 
62,000 projects from the PyPI basis. The style of the violations were recorded at the level of projects and modules, and from the results, discussed the adoption level of style in question.}
